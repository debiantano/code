\documentclass[a4paper]{article}

%%%%%%%%%%%%%%%%%%%%%%%%%%%%%%%%%%%%%%%%%%%%%%%%%
% idioma
\usepackage[utf8]{inputenc}
\usepackage[spanish]{babel}
% margenes del documento
\usepackage[margin=2cm, top=2cm, includefoot]{geometry}
% insercion de imágenes
\usepackage{graphicx}
% deteccion de color hexadecimal
\usepackage[table,xcdraw]{xcolor}
% insertar cuadros en la portada
\usepackage[most]{tcolorbox}
% definir el estilo de la página
\usepackage{fancyhdr}
% GESTION DE HIPERVINCULOS
\usepackage{hyperref}
% ARREGLO DE LA TABULACION
\usepackage{parskip}
% CAMBIAR NOMBRE CAPTION
\usepackage[figurename=FIG]{caption}
% REALIZAR DIAGRAMAS
\usepackage{smartdiagram}
% INSERCION DE CODIGO
\usepackage{listings}
% INSERCION DE TABLAS
\usepackage{zed-csp}

%-----------------
\definecolor{codegreen}{rgb}{0,0.6,0}
\definecolor{codegray}{rgb}{0.5,0.5,0.5}
\definecolor{codepurple}{rgb}{0.58,0,0.82}
\definecolor{backcolour}{rgb}{0.95,0.95,0.92}

\lstdefinestyle{mystyle}{
    backgroundcolor=\color{backcolour},   
    commentstyle=\color{codegreen},
    keywordstyle=\color{magenta},
    numberstyle=\tiny\color{codegray},
    stringstyle=\color{codepurple},
    basicstyle=\ttfamily\footnotesize,
    breakatwhitespace=false,         
    breaklines=true,                 
    captionpos=b,                    
    keepspaces=true,                 
    numbers=left,                    
    numbersep=5pt,                  
    showspaces=false,                
    showstringspaces=false,
    showtabs=false,                  
    tabsize=2
}

\lstset{style=mystyle}
%__________________

%%%%%%%%%%%%%%%%%%%%%%%%%%%%%%%%%%%%%%%%%%%%%%%%%

% DECLARACIÓN DE VARIABLES
\newcommand{\logoPortada}{REPORTS/parrot.jpg}
\newcommand{\machineName}{Parrot}
\newcommand{\startDate}{13 de marzo del 2021}
% DECLARACION DE COLORES
\definecolor{redPortada}{HTML}{FF0000}
% RENOMBRAR VARIABLES
\addto\captionsspanish{\renewcommand{\contentsname}{Contenido}}
% CAMBIAR LISTING CAPTION(de los codigos)
\renewcommand{\lstlistingname}{Código}

% CONFIGURANDO CABECERA
\setlength{\headheight}{40.2pt}
\pagestyle{fancy}
\fancyhf{}
\lhead{\includegraphics[width=1.6cm]{\logoPortada}}
\rhead{\includegraphics[height=1cm]{\logoPortada}}
% anchura de la barra horizontal
\renewcommand{\headrulewidth}{1.5pt}
\renewcommand{\headrule}{\hbox to\headwidth{\color{redPortada}\leaders\hrule height \headrulewidth\hfill}}



\begin{document}
    % numero de paginas
    \cfoot{\thepage}
    % CREACION DE PORTADA
    \begin{titlepage}
    \centering
    \includegraphics[width=0.2\textwidth]{\logoPortada}\par\vspace{1cm}
    
    {\scshape\large\textbf{Informe Técnico}} \vspace{2cm}
    
    {\Huge\bfseries\textcolor{redPortada}{Máquina \machineName}}
    \vfill
    \vspace{1cm}
    \includegraphics[width=\textwidth,height=10cm, keepaspectratio]{\logoPortada}\par\vspace{1cm}
    \vfill
    \begin{tcolorbox}[colback=red!5!white,colframe=red!75!black]
        \centering
        Este documento es confidencial y posee información sensible.\\No deberia ser impreso o compartido con terceras entidades.
    \end{tcolorbox}
    \vfill
    {\large \startDate\par}
    
    \vfill
    \end{titlepage}
    \vfill
    
    \clearpage
    
    %%%%%%%%%%%%%%%%%%%%%%%%%%%%%%%%%%%%%%%%%%%%%
    \tableofcontents
    \clearpage
    %%%%%%%%%%%%%%%%%%%%%%%%%%%%%%%%%%%%%%%%%%%%%
    
    \section{Antecedentes}
    El presente documento recoge los resultados obtenidos durante la fase de auditoría realizada a la máquina {\textbf\machineName} de la plataforma \href{https://hackthebox.eu}{\textbf{\color{blue}HackTheBox}}.
    
    \vspace{0.2cm}
    
    \begin{figure}[h]
        \centering
        \includegraphics[width=0.6\textwidth]{\logoPortada}
        \caption{Detalles de la máquina}
    \end{figure}
    
    \vspace{0.2cm}
    
    \begin{tcolorbox}[enhanced,attach boxed title to top center={yshift=-3mm,yshifttext=-1mm},
      colback=blue!5!white,colframe=blue!75!black,colbacktitle=redPortada!80!black,
      title=Dirección URL,fonttitle=\bfseries,
      boxed title style={size=small,colframe=red!50!black} ]
      \centering
      \href{https://hackthebox.eu/home/machines/profile/98}{\color{brown}{https://hackthebox.eu/home/machines/profile/98}}
\end{tcolorbox}
    
    \vspace{0.3cm}
    
    \section{Objetivos}
    Conocer el estado actual del servidor \textbf{\machineName}, enumerando posibles vectores de explotación y determinando el alcance e impacto que un atacante podría ocasionar sobre el sistema en producción.
    
    \subsection{Consideraciones}
    Una vez terminada la fase de auditoría, se llevará a cabo una fase de saneamientos y buenas prácticas con el objetivo de securizar el servidor y evitar ser víctimas de un futuro atáque en basee a los vectores explotados.
    
    \vspace{0.3cm}
    
    \begin{figure}[h]
        \begin{center}
            \smartdiagram[priority descriptive diagram]{
                Reconocimiento sobre el sistema,
                Detección de vulnerabilidades,
                Explotación de vulnerabilidades,
                Securización del sistema
            }
        \end{center}
        \caption{Flujo del trabajo}
    \end{figure}
    
    \clearpage
    
    \section{Análisis de vulnerabilidades}
    
    \subsection{Reconocimiento inicial}
    Se comenzó realizando un análisis inicial sobre el sistema, verificando que el sistema objetivo se encontrara accesible desde el segmento de red en el que se opera.
    
    \begin{figure}[h]
        \centering
        \includegraphics[width=0.7\textwidth]{\logoPortada}
        \caption{Reconocimiento inicial sobre el sistema}
    \end{figure}
    
    \vspace{0.2cm}

    Una vez localizado se realizó un escaneo a travéz de la herramienta \textbf{nmap} para la detección de puertos abiertos, obteniendo los siguientes resultados.
    
    \begin{figure}[h]
        \centering
        \includegraphics[width=0.8\textwidth]{\logoPortada}
        \caption{Reconocimiento con nmap}
    \end{figure}
    
    \clearpage
    
    Asímismo, con el objetivo de evitar falsos positivos se diseñó un script en  \textbf{bash} para enumerar posibles puertos adicionales que la herramienta nmap no llegara a detectar.
    
    \vspace{0.2cm}
    
    \begin{lstlisting}[language=Bash, caption= Script personalizado para la enumeracion de puertos]
#!/bin/bash
trap ctrl_c INT
function ctrl_c(){
        echo -e "\n${yellowColour}[*]${endColour}${greenColour} Saliendo .. ${endColour}"
        tput cnorm
        exit 0
}
    
tput civis; for port in $(seq 1 65535); do
        timeout 1 bash -c "echo > /dev/tcp/$1/$port" 2>/dev/null && echo Port $port - OPEN" &
        done; wait; tput cnorm
    \end{lstlisting}
    
    \vspace{0.3cm}
    
    A través de este script, fue posible detectar puertos adicionales abiertos.
    
    \begin{schema}{TCP}
    Puertos
    \where
    593, 1337
    \end{schema}
    
    Una vez finalizada la fase de enumeración de puertos, se detectaron los servicios y versiones que corrían bajo estos, representando a continuación los más significativos bajo las cuales fue posible explotar el sistema:
    
    \begin{figure}[h]
        \centering
        \makebox[\textwidth]{\includegraphics[width=0.9\paperwidth]{\logoPortada}}
        \caption{Enumaración de servicios y versiones}
        \label{fig:servicesResults}
    \end{figure}
    
    \vspace{0.2cm}
    
    Tal y como se aprecia en la figura \ref{fig:servicesResults} de la página \pageref{fig:servicesResults} es posible identificar que se trata de una máquina con \textbf{Directorio Activo} configurado.
    
    
    
    
\end{document}
