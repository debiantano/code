\documentclass[titlepage,12pt]{article}

\usepackage[utf8]{inputenc}
\usepackage[spanish]{babel}
\usepackage[left=2cm, right=2.5cm, bottom=2.5cm, top=2.5cm]{geometry}
\usepackage{graphicx}



\title{Offensive Security Wireless Professional \\ Exam Report}
\author{OSID : 5555}
\date{\today}

\begin{document}

\maketitle          
\tableofcontents    % crear indice
\newpage    % indice independiente

\section{intro}
La documentación del examen OSWP de Offensive Security contiene todos los esfuerzos que se realizaron para
aprobar el examen O ensive Security Wireless Professional. Este informe se calificará desde el punto de vista de
corrección y plenitud de todos los aspectos del examen. El propósito de este informe es asegurar que el
El estudiante tiene los conocimientos técnicos necesarios para aprobar las calificaciones de Seguridad O ensiva.
Certificación Wireless Professional. Debe completar la documentación de este examen en su totalidad y
incluir las siguientes secciones:

\begin{itemize}
    \item Tutorial de la metodología y esquema detallado de los pasos tomados
    \item Cada hallazgo con capturas de pantalla, recorridos y comandos de muestra incluidos
    \item Cualquier artículo adicional que no se haya incluido
\end{itemize}

\newpage    % salto de pagina

\section{STAGE 1}
\subsection{Clave de la red inalámbrica}
Proporcione el contenido de la clave de red inalámbrica ``ETAPA 1``.

\subsection{Captura de pantalla}
Proporcione al menos una captura de pantalla de la clave de red inalámbrica ``ETAPA 1`` descifrada con éxito

    \begin{figure}[h]
      \centering
        \includegraphics{img.png}
      \caption[width=\textwidth]{Marcado de posición de imágen}
      %\label{fig:ejemplo}
    \end{figure}


\subsection{Pasos}
Proporcione una descripción detallada de su metodología para obtener la clave de red inalámbrica ``ETAPA 1``. La
las medidas adoptadas deben poder seguirse fácilmente y ser reproducibles si es necesario.


\section{STAGE 2}

\subsection{Clave de la red inalámbrica}
\subsection{Captura de pantalla}
\subsection{Pasos}

\newpage

\section{STAGE 3}

\subsection{Clave de la red inalámbrica}
\subsection{Captura de pantalla}
\subsection{Pasos}

\newpage

\section{Articulos adicionales no mecionados en el reporte}
Esta sección se coloca para cualquier elemento adicional que no se mencionó en el informe general.

\end{document}
